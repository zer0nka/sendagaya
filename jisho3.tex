%\documentclass[10pt,b5paper]{article}
\documentclass[10pt,a4paper]{article} 

\usepackage{cmap}					% поиск в PDF
\usepackage[T2A]{fontenc}			% кодировка
\usepackage[utf8]{inputenc}			% кодировка исходного текста
\usepackage[english,russian]{babel}
\usepackage{CJKutf8}
\usepackage{tabularx}
\usepackage{fancyhdr}
\usepackage{longtable}
\usepackage{tabu}
\usepackage{caption}
\captionsetup{justification   = raggedright,
	singlelinecheck = false}

\usepackage{etoolbox}
\newcounter{magicrownumbers}
\newcommand\rownumber{\stepcounter{magicrownumbers}\arabic{magicrownumbers}}
\newcounter{lessonnumbers}
\newcommand\lessonnumbers{\stepcounter{lessonnumbers}\arabic{lessonnumbers}}
\setcounter{lessonnumbers}{34} 
%\usepackage[a5paper]{geometry}

\pagestyle{fancy}
\fancyhf{}
\cfoot{\thepage \begin{CJK}{UTF8}{min} ページ\end{CJK}}

\begin{document}
\begin{CJK}{UTF8}{min}
\section*{\centering{コミュニケーション日本語 3 言葉 \lessonnumbers課}}

% い形容詞 
\begin{longtabu} to \textwidth {
		|X[1,c,m]
		|X[1,r,m]
		|X[4,l,m]
		|X[9,l,m]
		|X[6,l,m]
		|X[9,l,m]
		|}
	% caption
	\caption*{い形容詞} \\
	
	% head
	\hline
	 & 
	\multicolumn{2}{l|}{ページ} & 
	ひらがな & 
	かんじ & 
	トランスレート  \\ \hline
	\endhead
	
	% rows
	\hline
	\rownumber & 4 & 練習A2 & みっともない &  &   \\ \hline 


\end{longtabu}

% な形容詞 
\begin{longtabu} to \textwidth {
		|X[1,c,m]
		|X[1,r,m]
		|X[4,l,m]
		|X[9,l,m]
		|X[6,l,m]
		|X[9,l,m]
		|}
	% caption
	\caption*{い形容詞} \\
	
	% head
	\hline
	& 
	\multicolumn{2}{l|}{ページ} & 
	ひらがな & 
	かんじ & 
	トランスレート  \\ \hline
	\endhead
	
	% rows
	\hline
	\rownumber & 1 & 会話 & りっぱ・な & 立派 &   \\ \hline
	
\end{longtabu}

% 動詞 
\begin{longtabu} to \textwidth {
		|X[1,c,m]
		|X[1,r,m]
		|X[4,l,m]
		|X[9,l,m]
		|X[6,l,m]
		|X[9,l,m]
		|}
	% caption
	\caption*{動詞} \\
	
	% head
	\hline
	& 
	\multicolumn{2}{l|}{ページ} & 
	ひらがな & 
	かんじ & 
	トランスレート  \\ \hline
	\endhead
	
	% rows
	\hline
	\rownumber & 1 & 会話 & うつる & 映る &   \\ \hline 
	\rownumber & 1 & 会話 & ちる & 散る &   \\ \hline 
	\rownumber & 3 & 例文1 & (胃が)もたれる &  &   \\ \hline 
	\rownumber & 4 & 練習A2 & のびる & 伸びる &   \\ \hline 
	\rownumber & 4 & 練習A1 & みつける & 見つける &   \\ \hline 
	\rownumber & 6 & 練習B2 & ていしゅつ(する) & 提出 &   \\ \hline 
%	\rownumber &  &  &  &  &   \\ \hline

	
\end{longtabu}

% 名詞 
\begin{longtabu} to \textwidth {
		|X[1,c,m]
		|X[1,r,m]
		|X[4,l,m]
		|X[9,l,m]
		|X[6,l,m]
		|X[9,l,m]
		|}
	% caption
	\caption*{名詞} \\
	
	% head
	\hline
	& 
	\multicolumn{2}{l|}{ページ} & 
	ひらがな & 
	かんじ & 
	トランスレート  \\ \hline
	\endhead
	
	% rows
	\hline
	\rownumber & 1 & 会話 & かず & 数 &   \\ \hline
	\rownumber & 1 & 会話 & おほり & お堀 &   \\ \hline
	\rownumber & 1 & 会話 & いしがき & 石垣 &   \\ \hline
	\rownumber & 1 & 会話 & まんかい & 満開 &   \\ \hline
	\rownumber & 1 & 会話 & おおあめ & 大雨 &   \\ \hline
	\rownumber & 1 & 会話 & さいこう & 最高 &   \\ \hline
	\rownumber & 3 & 会話 & い & 胃 &   \\ \hline
	\rownumber & 3 & 例文1 & さむけ & 寒気 & холод  \\ \hline
	\rownumber & 4 & 練習A1 & きょうみ & 興味 &   \\ \hline
	\rownumber & 4 & 練習A2 & うわぎ & 上着 &   \\ \hline
	\rownumber & 4 & 練習A2 & ゆうしょく & 夕食 &   \\ \hline
	\rownumber & 5 & 練習A3 & さくぶん & 作文 &   \\ \hline
	\rownumber & 5 & 練習A3 & いしゃ & 医者 &   \\ \hline
	\rownumber & 5 & 練習A3 & おいしゃさん & お医者さん &   \\ \hline
	\rownumber & 5 & 練習A3 & えいよう & 栄養 &   \\ \hline
	\rownumber & 5 & 練習A3 & バランス &  &   \\ \hline
	\rownumber & 5 & 練習A3 & おおや & 大家 &   \\ \hline
	\rownumber & 5 & 練習A3 & ファックス & FAX &   \\ \hline
	\rownumber & 5 & 練習A3 & てんちょう & 店長 &   \\ \hline
	\rownumber & 6 & 練習B2 & がんしょ & 願書 &   \\ \hline
%	\rownumber &  &  &  &  &   \\ \hline
	
\end{longtabu}

\begin{longtabu} to \textwidth {
		|X[1,c,m]
		|X[1,r,m]
		|X[4,l,m]
		|X[9,l,m]
		|X[6,l,m]
		|X[9,l,m]
		|}
	% caption
	\caption*{固有名詞} \\
	
	% head
	\hline
	& 
	\multicolumn{2}{l|}{ページ} & 
	ひらがな & 
	かんじ & 
	トランスレート  \\ \hline
	\endhead
	
	% rows
	\hline
	\rownumber & 6 & 練習B2 & ふじわら & 藤原 &   \\ \hline
%	\rownumber &  &  &  &  &   \\ \hline
	
\end{longtabu}

\begin{longtabu} to \textwidth {
		|X[1,c,m]
		|X[1,r,m]
		|X[4,l,m]
		|X[9,l,m]
		|X[6,l,m]
		|X[9,l,m]
		|}
	% caption
	\caption*{表現など} \\
	
	% head
	\hline
	& 
	\multicolumn{2}{l|}{ページ} & 
	ひらがな & 
	かんじ & 
	トランスレート  \\ \hline
	\endhead
	
	% rows
	\hline
	\rownumber & 3 & 例文1 & さきに & 先に &   \\ \hline
	\rownumber & 6 & 練習B1 & 〜ご & 後 &   \\ \hline
%	\rownumber &  &  &  &  &   \\ \hline
	
\end{longtabu}

% clean table counter and start new page
\preto\tabular{\setcounter{magicrownumbers}{0}}
\newpage

\section*{\centering{コミュニケーション日本語 3 言葉 \lessonnumbers課}}
\end{CJK}
\end{document}